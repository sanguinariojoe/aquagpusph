\chapter{Running \NAME}
\label{s:running}
%
\section{Launching simulation}
\label{ss:running:launching}
%
In order to launch \NAME simulation you need to execute the program
specifying the XML definition file in a command line. If you have
installed \NAME in a folder present on the PATH environment variable
you can launch a 3D simulation typing:
%
\begin{alltt}
AQUAgpusph -i \emph{PathToXML}
\end{alltt}
%
or
%
\begin{alltt}
AQUAgpusph2D -i \emph{PathToXML}
\end{alltt}
%
for 2D simulations.\rc
%
\NAME accepts several command line options. You can get a complete list
and description of them in the help page (`-h/--help' help option). If
the `-i/--input' option is not provided, Input.xml file will be selected
as the XML case definition file, so must exist in the execution folder.
You can consider use '-n/--no-reassembly' since output files reassembly
can be a really time and hard-disk consuming operation, and simply
unuseful if the simulation has been ran without interruptions.\rc
%
You can start a simulation from previous output file. If you plan to perform
the simulation in several runs, consider not perform the reassembly before
the last execution. Also you will need to set options according (see section
\ref{sss:XML:Settings} to learn more about this).\rc
%
\NAME execution can be stopped using `c' key. When `c' key event is detected
simulation will be stopped at the end of the active time step, and output
saved and closed correctly.\rc
%
Eventually you can stop the simulation pressing `Ctrl'+`c', but then the
simulation will stop immediately, and can result in output unreadable
files\footnote{H5Part output will be surely broken}.
%
\section{Tracking simulation}
\label{ss:running:tracking}
%
\subsection{General}
%
\NAME prints the log information in 2 different ways, on the screen and in a
log HTML file. Both of them have similar content, and can be controlled
independently.\rc
%
\subsection{Screen log information}
\label{sss:running:screenlog}
%
The screen real time output can be controlled in the settings configuration
section, see chapter \ref{sss:XML:Settings} for details.\rc
%
The screen output will report some data organized in groups:
%
\begin{itemize}
	\item Simulation progress
	\begin{itemize}
		\item Simulation time $[\mbox{s}]$.
		\item Simulation time step $[\mbox{s}]$.
		\item Percentage of the simulation accomplished.
		\item Last output frame printed.
		\item Estimated Time to Arrive $[\mbox{s}]$.
	\end{itemize}
	\item Profile data (Only Debug version)
	\item Calculation server memory
	\begin{itemize}
		\item Number of particles.
		\item Number of cells.
		\item Allocated memory $[\mbox{bytes}]$.
	\end{itemize}	
	\item Energy data
	\begin{itemize}
		\item Momentums $[\mbox{N} \cdot \mbox{s}]$
		\item Angular momentums $[\mbox{N} \cdot \mbox{m s}]$
		\item Kinetic energy $[\mbox{J}]$
		\item Potential energy $[\mbox{J}]$
		\item Total energy $[\mbox{J}]$
	\end{itemize}	
	\item Events registry
\end{itemize}
%
Depending on your settings the ``Energy data'' group will be printed
more often, or eventually not printed. Actually energy data is not
computed on GPU, which implies a significant time consuming process
and it's strongly recommended to decrease print rate as much as possible.\rc
%
In the other hand, when \NAME has computed the energy data in a certain
time step, it does not recompute it until new step is called, so you can
consider to set that all energy data outputs to be performed on the same
instant (see the sections \ref{sss:running:logfile} and
\ref{ss:running:energyoutput} to learn more about this).\rc
%
\NAME also shows in real time an events registry, where some relevant
messages are shown. Messages are classified in 4
categories\footnote{Colors are only actives if \NAME has been built with
NCurses support, see chapter \ref{s:install} for details.}:
%
\begin{enumerate}
	\item \textbf{Terminal output}: Initialization and closing information.
	\item \textbf{White}: Relevant information.
	\item \textbf{Orange}: Warning messages.
	\item \textbf{Red}: Error messages.
\end{enumerate}
%
In section \ref{sss:running:messages} you can find a list of the messages
that you can expect from the simulation, and the possible causes of them.
%
\subsection{HTML log file}
\label{sss:running:logfile}
%
Depending on the settings discussed on the section \ref{sss:XML:Timing},
\NAME can be configured to print a log HTML file ``Log.html''\footnote{If
prefix is not set as command line option. You can set a prefix that will
be inserted at the start of all the output file names}. The log file contains
useful information to to track it remotely and in order to know and
preserve all incidents and remarks associated with the simulation.\rc
%
HTML log file contains information classified in 3 categories:
%
\begin{enumerate}
	\item \textbf{Black}: Relevant information.
	\item \textbf{Orange}: Warning messages.
	\item \textbf{Red}: Error messages.
\end{enumerate}
%
In section \ref{sss:running:messages} you can find a list with some relevant
messages that you can get on the log HTML file. Also Information about date
and time, and printed output files will be included on this file.
%
\subsection{Messages glossary}
\label{sss:running:messages}
%
\NAME reports several messages on real time. In this section we show all
the messages that can be shown, with description, source, and possible
solution if proceed.\rc
%
The messages have the following structure:
%
\begin{alltt}
[\emph{TYPE} (T=\emph{TIME}s, Step=\emph{STEP})] (\emph{METHOD}): \emph{MESSAGE}
\end{alltt}
%
\textit{TYPE} is the type of message, that can be INFO, WARNING, or ERROR;
\textit{TIME} is the simulation time where the event has been registered;
\textit{STEP} is the simulation time step where the event has been registered;
\textit{METHOD} is the \NAME routine that registered the event; \textit{MESSAGE}
is the description of the event.\rc
%
The list of events that you can obtain during the \NAME execution are:
%
\subsubsection{$[$ERROR$]$ VTK output called, but not supported:}
%
If \NAME has been built without VTK support, but VTK files output have
been requested, this error will be reported each time that output is
called. See chapter \ref{sss:install:cmake} in order to know how to
activate support for VTK files.
%
\subsubsection{$[$ERROR$]$ H5Part output called, but not supported:}
%
If \NAME has been built without H5Part support, but H5Part files output
have been requested, this error will be reported each time that output
is called. See chapter \ref{sss:install:cmake} in order to know how to
activate support for H5Part files.
%
\subsubsection{$[$ERROR$]$ Can't send variable to kernel:}
%
Error produced when \NAME can't send a variable to an OpenCL kernel. If
you modified an OpenCL kernel and/or the source code to customize \NAME
functionality, ensure that sent variables from \NAME and declared ones
on OpenCL kernel matchs.\rc
This error will stop \NAME execution.
%
\subsubsection{$[$ERROR$]$ Can't execute the kernel:}
%
OpenCL kernel can't be launched. If the problem is specifically documented,
it will be added to the log too. Common causes of this problem is a custom
OpenCL kernel, or unallocatable resources.\rc
This error will stop the \NAME execution.
%
\subsubsection{$[$ERROR$]$ Can't wait to kernels end:}
%
If \NAME has been built in Debug mode, an execution profile will be activated in
order to report time consumed by each stage of the code. This error is reported when
program can't wait for kernel execution, usually caused by a kernel execution error.\rc
This error will stop the \NAME execution.
%
\subsubsection{$[$ERROR$]$ Can't profile kernel execution:}
%
Similar to previous one, in this case program can wait to kernel finish, but can not be
profiled (execution time can not be extracted).\rc
This error will stop the \NAME execution.
%
\subsubsection{$[$ERROR$]$ Resultant keys overflows unsigned int type:}
%
Radix sort stage related error. This error is usually caused by a too large number
of cells, that can happen if a domain is not bounded (see section \ref{sss:XML:SPH} in order t
know how to set it) and one or more particles go out of physical boundaries.\rc
This error will stop the \NAME execution.
%
\subsubsection{$[$ERROR$]$ perform() execution fail:}
%
Error associated to simulations with Python script motion control. The error is
caused by a Python runtime execution error detected at perform() required method.
Motion control Python script must be fixed.\rc
Check chapter \ref{sss:XML:Movements} in order to learn more about how to set a Python script
controlled motion, and \ref{sss:aquagpusph:motions:Controls} to learn more about the Python scripting.\rc 
This error will stop the \NAME execution.
%
\subsubsection{$[$ERROR$]$ perform() returned quaternion is not valid:}
%
Error associated to simulations with Python script motion control. The error is
caused by a wrong returned value by Python script perform() method.
Motion control Python script must be fixed.\rc
You can see chapter \ref{sss:XML:Movements} in order to learn more about how to set a Python script
controlled motion, and \ref{sss:aquagpusph:motions:Controls} to learn more about the Python scripting.\rc 
This error will stop the \NAME execution.
%
\subsubsection{$[$ERROR$]$ Failure retrieving memory from server:}
%
OpenCL memory can not be recovered on host. \NAME eventually provide details
of this error into the log.\rc
This error will stop the \NAME execution.
%
\subsubsection{$[$ERROR$]$ Failure sending memory to server:}
%
Data can not be sent from host to OpenCL platform. \NAME eventually provide details
of this error into the log.\rc
This error will stop the \NAME execution.
%
\subsubsection{$[$ERROR$]$ Fail allocating memory for $array$ ($m$ bytes):}
%
Produced by an unacceptable number of cells number. Requested memory on OpenCL
platform can't be allocated.\rc
This error will stop the \NAME execution.
%
\subsubsection{$[$ERROR$]$ timestep has dramaticaly decreased! [$dt_t$ -$>$ $dt_{t+dt}$]:}
%
In \NAME a variable time step can be set, see chapter \ref{sss:XML:Timing} to learn more
about the time step alternatives. Variable time step is a good way to walk around
some transitional instabilities, decreasing instantaneously the time step,
and trying to avoid that the particles pass trough walls therefore.\rc
This error is reported when the time step is reduced in one or more orders of
magnitude. If this error is reported the simulation can continue, but long
computational times with wrong results can be expected, and you may
consider the possibility of stop and fix simulation in order to relaunch it.
%
\subsubsection{$[$WARNING$]$ timestep changed [$dt_t$ -$>$ $dt_{t+dt}$]:}
%
In \NAME a variable time step can be set, see chapter \ref{sss:XML:Timing} to learn more
about the time step alternatives. Variable time step is a good way to walk around
some transitional instabilities, decreasing instantaneously the time step, and
trying to avoid that the particles pass trough walls therefore.\rc
This message indicates that the time step has been adapted. The time step variation
is an indicative that the simulation is not running right, so if you receive
too much time step change reports, or the time step change is large, maybe you
must consider repeat the simulation increasing the sound speed or the time step
divisor.
%
\subsubsection{$[$WARNING$]$ Number of cells increased [$n_t$ -$>$ $n_{t+dt}$]:}
%
Like other SPH codes, \NAME neighbour particles localization is based on a
PIC algorithm, where each particle is marked by a grid cell where is situated. The
number of cells where the particles can be located depends on the extension of
fluid domain. Therefore, it can be modified along the simulation. You can learn more about
cells usage on \NAME at chapter \ref{ss:aquagpusph:linklist}.\rc
In most simulations fluid bounds are not constant along the simulation, so some
reports about cell numbers increase can occur. If this increase is too large the reallocation of
memory can be impossible, breaking the simulation, and possibly indicating that some
particles has abandoned the physical domain.
%

%
\subsubsection{$[$INFO$]$ List too small, local memory usage will be avoided:}
%
When the number of particles is too small, local memory can't be used in the transpose
stage of the Radix sort process (See the chapter \ref{s:aquagpusph}). \NAME will run more slowly
therefore, but since the number of particles is not large probably you must can ignore
this notification.\rc
Sometimes you can try to increase sightly the number of particles in order to recover
the possibility of using local memory, increasing the resolution and the performance.
%
\subsubsection{$[$INFO$]$ Interrumption detected:}
%
Notice that 'c' key has been pressed, and the simulation will stop.
%
\subsubsection{$[$INFO$]$ Allocated memory = $m$ bytes:}
%
Reports allocated memory on OpenCL platform.
%

%
