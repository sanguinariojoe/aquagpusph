\chapter{Directory tree}
%
Following all the files provided with \NAME package, with its location, is 
shown.
%
The files autogenerated during the configuration, build and install processes 
are not shown.
%
The purpose of some of these files and folders is explained later.

\vspace{0.5cm}
\rule{\textwidth}{1pt}
\vspace{0.5cm}
\dirtree{%
.1 { \includegraphics[width=0.05\textwidth]{tree/Actions-document-open-folder-icon} aquagpusph-1.5 }.
.2 { \includegraphics[width=0.05\textwidth]{tree/Mimetypes-text-plain-icon} Releases notes }.
.2 { \includegraphics[width=0.05\textwidth]{tree/Mimetypes-text-plain-icon} TODO }.
.2 { \includegraphics[width=0.05\textwidth]{tree/Mimetypes-x-office-document-icon} LICENSE }.
.2 { \includegraphics[width=0.05\textwidth]{tree/Mimetypes-text-plain-icon} CMakeLists.txt }.
.2 { \includegraphics[width=0.05\textwidth]{tree/Mimetypes-application-x-shellscript-icon} countLines.sh }.
.2 { \includegraphics[width=0.05\textwidth]{tree/Mimetypes-text-plain-icon} Code styling }.
.2 { \includegraphics[width=0.05\textwidth]{tree/Mimetypes-text-plain-icon} config.h.cmake }.
.2 { \includegraphics[width=0.05\textwidth]{tree/Mimetypes-text-plain-icon} README.md }.
.2 { \includegraphics[width=0.05\textwidth]{tree/Actions-document-open-folder-icon} doc }.
.3 { \includegraphics[width=0.05\textwidth]{tree/Mimetypes-application-pdf-icon} opencl-1-1-quick-reference-card.pdf }.
.3 { \includegraphics[width=0.05\textwidth]{tree/Mimetypes-text-plain-icon} CMakeLists.txt }.
.3 { \includegraphics[width=0.05\textwidth]{tree/Mimetypes-application-pdf-icon} Doxygen Quick Reference.pdf }.
.3 { \includegraphics[width=0.05\textwidth]{tree/Actions-document-open-folder-icon} cMake }.
.4 { \includegraphics[width=0.05\textwidth]{tree/Mimetypes-x-office-document-icon} Doxyfile.in }.
.3 { \includegraphics[width=0.05\textwidth]{tree/Actions-document-open-folder-icon} devManual }.
.4 { \includegraphics[width=0.05\textwidth]{tree/Mimetypes-text-x-bibtex-icon} directory\_tree.tex }.
.4 { \includegraphics[width=0.05\textwidth]{tree/Mimetypes-x-office-document-icon} main.aux }.
.4 { \includegraphics[width=0.05\textwidth]{tree/Mimetypes-x-office-document-icon} main.out }.
.4 { \includegraphics[width=0.05\textwidth]{tree/Mimetypes-text-x-bibtex-icon} structure.tex }.
.4 { \includegraphics[width=0.05\textwidth]{tree/Mimetypes-text-x-bibtex-icon} abstract.tex }.
.4 { \includegraphics[width=0.05\textwidth]{tree/Mimetypes-text-plain-icon} main.log }.
.4 { \includegraphics[width=0.05\textwidth]{tree/Mimetypes-application-x-shellscript-icon} buildPDF.sh }.
.4 { \includegraphics[width=0.05\textwidth]{tree/Mimetypes-x-office-document-icon} main.toc }.
.4 { \includegraphics[width=0.05\textwidth]{tree/Mimetypes-text-x-bibtex-icon} main.tex }.
.4 { \includegraphics[width=0.05\textwidth]{tree/Mimetypes-text-x-bibtex-icon} tree\_py.tex }.
.3 { \includegraphics[width=0.05\textwidth]{tree/Actions-document-open-folder-icon} userManual }.
.4 { \includegraphics[width=0.05\textwidth]{tree/Mimetypes-text-x-bibtex-icon} running.tex }.
.4 { \includegraphics[width=0.05\textwidth]{tree/Mimetypes-text-x-bibtex-icon} lateral\_water\_1x\_ghost.tex }.
.4 { \includegraphics[width=0.05\textwidth]{tree/Mimetypes-text-x-bibtex-icon} warningMsg.tex }.
.4 { \includegraphics[width=0.05\textwidth]{tree/Mimetypes-text-x-bibtex-icon} elasticbounce.tex }.
.4 { \includegraphics[width=0.05\textwidth]{tree/Mimetypes-text-x-bibtex-icon} errorMsg.tex }.
.4 { \includegraphics[width=0.05\textwidth]{tree/Mimetypes-text-x-bibtex-icon} directory\_tree.tex }.
.4 { \includegraphics[width=0.05\textwidth]{tree/Mimetypes-text-x-bibtex-icon} examples.tex }.
.4 { \includegraphics[width=0.05\textwidth]{tree/Mimetypes-text-x-bibtex-icon} asciiPartsInput.tex }.
.4 { \includegraphics[width=0.05\textwidth]{tree/Mimetypes-text-x-bibtex-icon} motionTypes.tex }.
.4 { \includegraphics[width=0.05\textwidth]{tree/Mimetypes-text-x-bibtex-icon} paraview.tex }.
.4 { \includegraphics[width=0.05\textwidth]{tree/Mimetypes-text-plain-icon} texput.log }.
.4 { \includegraphics[width=0.05\textwidth]{tree/Mimetypes-text-x-bibtex-icon} sensors.tex }.
.4 { \includegraphics[width=0.05\textwidth]{tree/Mimetypes-text-x-bibtex-icon} DeLeffe.tex }.
.4 { \includegraphics[width=0.05\textwidth]{tree/Mimetypes-text-x-bibtex-icon} infoMsg.tex }.
.4 { \includegraphics[width=0.05\textwidth]{tree/Mimetypes-text-x-bibtex-icon} ghostparticles.tex }.
.4 { \includegraphics[width=0.05\textwidth]{tree/Mimetypes-text-x-bibtex-icon} partsInput.tex }.
.4 { \includegraphics[width=0.05\textwidth]{tree/Mimetypes-text-x-bibtex-icon} aquagpusph.tex }.
.4 { \includegraphics[width=0.05\textwidth]{tree/Mimetypes-text-x-bibtex-icon} abstract.tex }.
.4 { \includegraphics[width=0.05\textwidth]{tree/Mimetypes-text-x-bibtex-icon} xmlPartsInput.tex }.
.4 { \includegraphics[width=0.05\textwidth]{tree/Mimetypes-text-x-bibtex-icon} lateral\_water\_1x\_deleffe.tex }.
.4 { \includegraphics[width=0.05\textwidth]{tree/Mimetypes-application-x-shellscript-icon} buildPDF.sh }.
.4 { \includegraphics[width=0.05\textwidth]{tree/Mimetypes-x-office-document-icon} bib.bib }.
.4 { \includegraphics[width=0.05\textwidth]{tree/Mimetypes-application-pdf-icon} aquagpusph-usermanual.pdf }.
.4 { \includegraphics[width=0.05\textwidth]{tree/Mimetypes-text-x-bibtex-icon} physics.tex }.
.4 { \includegraphics[width=0.05\textwidth]{tree/Mimetypes-text-x-bibtex-icon} install.tex }.
.4 { \includegraphics[width=0.05\textwidth]{tree/Mimetypes-text-x-bibtex-icon} fixparticles.tex }.
.4 { \includegraphics[width=0.05\textwidth]{tree/Mimetypes-text-x-bibtex-icon} outputFiles.tex }.
.4 { \includegraphics[width=0.05\textwidth]{tree/Mimetypes-text-x-bibtex-icon} main.tex }.
.4 { \includegraphics[width=0.05\textwidth]{tree/Mimetypes-text-x-bibtex-icon} gidPartsInput.tex }.
.4 { \includegraphics[width=0.05\textwidth]{tree/Mimetypes-text-x-bibtex-icon} tree\_py.tex }.
.4 { \includegraphics[width=0.05\textwidth]{tree/Mimetypes-text-x-bibtex-icon} xmlInput.tex }.
.4 { \includegraphics[width=0.05\textwidth]{tree/Mimetypes-text-x-bibtex-icon} motionControls.tex }.
.4 { \includegraphics[width=0.05\textwidth]{tree/Actions-document-open-folder-icon} images }.
.5 { \includegraphics[width=0.05\textwidth]{tree/Mimetypes-application-x-egon-icon} GhostParticlesCorner.svg }.
.5 { \includegraphics[width=0.05\textwidth]{tree/Mimetypes-application-x-egon-icon} FixedParticles.png }.
.5 { \includegraphics[width=0.05\textwidth]{tree/Mimetypes-application-x-egon-icon} DeLeffeCorner.svg }.
.5 { \includegraphics[width=0.05\textwidth]{tree/Mimetypes-application-x-egon-icon} GhostParticles.svg }.
.5 { \includegraphics[width=0.05\textwidth]{tree/Mimetypes-application-x-egon-icon} SPHInterpolation.png }.
.5 { \includegraphics[width=0.05\textwidth]{tree/Mimetypes-application-x-egon-icon} FixedParticles.svg }.
.5 { \includegraphics[width=0.05\textwidth]{tree/Mimetypes-application-x-egon-icon} ElasticBounce.svg }.
.5 { \includegraphics[width=0.05\textwidth]{tree/Mimetypes-application-x-egon-icon} GhostParticles.png }.
.5 { \includegraphics[width=0.05\textwidth]{tree/Mimetypes-application-x-egon-icon} LinkList.png }.
.5 { \includegraphics[width=0.05\textwidth]{tree/Mimetypes-application-x-egon-icon} ElasticBounce.png }.
.5 { \includegraphics[width=0.05\textwidth]{tree/Mimetypes-application-x-egon-icon} BC.png }.
.5 { \includegraphics[width=0.05\textwidth]{tree/Mimetypes-application-x-egon-icon} CoalescenceRead.png }.
.5 { \includegraphics[width=0.05\textwidth]{tree/Mimetypes-application-x-egon-icon} DeLeffe.svg }.
.5 { \includegraphics[width=0.05\textwidth]{tree/Mimetypes-application-x-egon-icon} DeLeffeCorner.png }.
.5 { \includegraphics[width=0.05\textwidth]{tree/Mimetypes-application-x-egon-icon} GhostParticlesCorner.png }.
.5 { \includegraphics[width=0.05\textwidth]{tree/Mimetypes-application-x-egon-icon} BC.svg }.
.5 { \includegraphics[width=0.05\textwidth]{tree/Mimetypes-application-x-egon-icon} SPHInterpolation.svg }.
.5 { \includegraphics[width=0.05\textwidth]{tree/Mimetypes-application-x-egon-icon} LinkList.svg }.
.5 { \includegraphics[width=0.05\textwidth]{tree/Mimetypes-application-x-egon-icon} DeLeffe.png }.
.5 { \includegraphics[width=0.05\textwidth]{tree/Mimetypes-application-x-egon-icon} CC\_88x31.png }.
.5 { \includegraphics[width=0.05\textwidth]{tree/Mimetypes-x-office-document-icon} Diagrams.odg }.
.5 { \includegraphics[width=0.05\textwidth]{tree/Mimetypes-application-x-egon-icon} GeneralDiagram.png }.
.5 { \includegraphics[width=0.05\textwidth]{tree/Actions-document-open-folder-icon} lateral\_water\_1x\_deleffe }.
.6 { \includegraphics[width=0.05\textwidth]{tree/Mimetypes-application-x-egon-icon} paraviewfluid3D.jpg }.
.6 { \includegraphics[width=0.05\textwidth]{tree/Mimetypes-application-x-egon-icon} animation\_freeslip.jpg }.
.6 { \includegraphics[width=0.05\textwidth]{tree/Mimetypes-application-x-egon-icon} particles.png }.
.6 { \includegraphics[width=0.05\textwidth]{tree/Mimetypes-application-x-egon-icon} animation\_noslip.jpg }.
.6 { \includegraphics[width=0.05\textwidth]{tree/Mimetypes-application-pdf-icon} press.pdf }.
.6 { \includegraphics[width=0.05\textwidth]{tree/Mimetypes-application-x-egon-icon} paraviewcolormap.jpg }.
.6 { \includegraphics[width=0.05\textwidth]{tree/Mimetypes-application-x-egon-icon} paraviewfilter.jpg }.
.6 { \includegraphics[width=0.05\textwidth]{tree/Mimetypes-application-x-egon-icon} tank.png }.
.6 { \includegraphics[width=0.05\textwidth]{tree/Mimetypes-application-x-egon-icon} sensors\_noslip.png }.
.6 { \includegraphics[width=0.05\textwidth]{tree/Mimetypes-application-x-egon-icon} particles.svg }.
.6 { \includegraphics[width=0.05\textwidth]{tree/Mimetypes-application-x-egon-icon} paraviewfilteroptions.jpg }.
.6 { \includegraphics[width=0.05\textwidth]{tree/Mimetypes-application-x-egon-icon} paraviewsolid3D.jpg }.
.6 { \includegraphics[width=0.05\textwidth]{tree/Mimetypes-application-x-egon-icon} sensors\_freeslip.png }.
.6 { \includegraphics[width=0.05\textwidth]{tree/Mimetypes-application-x-egon-icon} paraviewsolidfilter.jpg }.
.6 { \includegraphics[width=0.05\textwidth]{tree/Mimetypes-x-office-document-icon} lateral\_water\_1x.gnuplot }.
.5 { \includegraphics[width=0.05\textwidth]{tree/Actions-document-open-folder-icon} wendland }.
.6 { \includegraphics[width=0.05\textwidth]{tree/Mimetypes-application-pdf-icon} wendland2D.pdf }.
.6 { \includegraphics[width=0.05\textwidth]{tree/Mimetypes-text-x-python-icon} wendland2D.py }.
.5 { \includegraphics[width=0.05\textwidth]{tree/Actions-document-open-folder-icon} tree }.
.6 { \includegraphics[width=0.05\textwidth]{tree/Mimetypes-application-x-egon-icon} Mimetypes-application-x-shellscript-icon.png }.
.6 { \includegraphics[width=0.05\textwidth]{tree/Mimetypes-application-x-egon-icon} Mimetypes-text-x-bibtex-icon.png }.
.6 { \includegraphics[width=0.05\textwidth]{tree/Mimetypes-application-x-egon-icon} Mimetypes-application-vnd-oasis-opendocument-spreadsheet-icon.png }.
.6 { \includegraphics[width=0.05\textwidth]{tree/Mimetypes-application-x-egon-icon} Mimetypes-application-xml-icon.png }.
.6 { \includegraphics[width=0.05\textwidth]{tree/Mimetypes-application-x-egon-icon} Mimetypes-application-x-egon-icon.png }.
.6 { \includegraphics[width=0.05\textwidth]{tree/Mimetypes-application-x-egon-icon} Mimetypes-application-x-archive-icon.png }.
.6 { \includegraphics[width=0.05\textwidth]{tree/Mimetypes-application-x-egon-icon} Actions-edit-select-all-icon.png }.
.6 { \includegraphics[width=0.05\textwidth]{tree/Mimetypes-application-x-egon-icon} Mimetypes-application-rtf-icon.png }.
.6 { \includegraphics[width=0.05\textwidth]{tree/Mimetypes-application-x-egon-icon} Mimetypes-text-x-c-plus-plus-src-icon.png }.
.6 { \includegraphics[width=0.05\textwidth]{tree/Mimetypes-text-x-python-icon} tree.py }.
.6 { \includegraphics[width=0.05\textwidth]{tree/Mimetypes-application-x-egon-icon} Mimetypes-text-plain-icon.png }.
.6 { \includegraphics[width=0.05\textwidth]{tree/Mimetypes-application-x-egon-icon} Mimetypes-text-x-csrc-icon.png }.
.6 { \includegraphics[width=0.05\textwidth]{tree/Mimetypes-application-x-egon-icon} Mimetypes-text-x-python-icon.png }.
.6 { \includegraphics[width=0.05\textwidth]{tree/Mimetypes-application-x-egon-icon} Mimetypes-application-x-applix-word-icon.png }.
.6 { \includegraphics[width=0.05\textwidth]{tree/Mimetypes-application-x-egon-icon} Mimetypes-application-x-desktop-icon.png }.
.6 { \includegraphics[width=0.05\textwidth]{tree/Mimetypes-application-x-egon-icon} Mimetypes-x-office-document-icon.png }.
.6 { \includegraphics[width=0.05\textwidth]{tree/Mimetypes-application-x-egon-icon} Mimetypes-application-vnd-oasis-opendocument-graphics-icon.png }.
.6 { \includegraphics[width=0.05\textwidth]{tree/Mimetypes-application-x-egon-icon} Mimetypes-application-vnd-stardivision-draw-icon.png }.
.6 { \includegraphics[width=0.05\textwidth]{tree/Mimetypes-application-x-egon-icon} Mimetypes-application-pdf-icon.png }.
.6 { \includegraphics[width=0.05\textwidth]{tree/Mimetypes-application-x-egon-icon} Mimetypes-application-octet-stream-icon.png }.
.6 { \includegraphics[width=0.05\textwidth]{tree/Mimetypes-application-x-egon-icon} Mimetypes-text-x-chdr-icon.png }.
.6 { \includegraphics[width=0.05\textwidth]{tree/Mimetypes-application-x-egon-icon} Actions-document-open-folder-icon.png }.
.5 { \includegraphics[width=0.05\textwidth]{tree/Actions-document-open-folder-icon} lateral\_water\_1x\_ghost }.
.6 { \includegraphics[width=0.05\textwidth]{tree/Mimetypes-application-x-egon-icon} animation\_freeslip.jpg }.
.6 { \includegraphics[width=0.05\textwidth]{tree/Mimetypes-application-x-egon-icon} sensors\_noslip.png }.
.6 { \includegraphics[width=0.05\textwidth]{tree/Mimetypes-application-x-egon-icon} sensors\_freeslip.png }.
.2 { \includegraphics[width=0.05\textwidth]{tree/Actions-document-open-folder-icon} tools }.
.3 { \includegraphics[width=0.05\textwidth]{tree/Mimetypes-text-x-python-icon} setup.py }.
.3 { \includegraphics[width=0.05\textwidth]{tree/Mimetypes-text-plain-icon} CMakeLists.txt }.
.3 { \includegraphics[width=0.05\textwidth]{tree/Actions-document-open-folder-icon} aquagpusph\_postprocessing }.
.4 { \includegraphics[width=0.05\textwidth]{tree/Mimetypes-x-office-document-icon} vtk-freesurface }.
.4 { \includegraphics[width=0.05\textwidth]{tree/Mimetypes-x-office-document-icon} pvd-locale }.
.3 { \includegraphics[width=0.05\textwidth]{tree/Actions-document-open-folder-icon} aquagpusph\_preprocessing }.
.4 { \includegraphics[width=0.05\textwidth]{tree/Mimetypes-x-office-document-icon} AQUAgpusph-loadGiD }.
.4 { \includegraphics[width=0.05\textwidth]{tree/Mimetypes-x-office-document-icon} AQUAgpusph-loadAbaqus }.
.4 { \includegraphics[width=0.05\textwidth]{tree/Mimetypes-text-x-python-icon} \_\_init\_\_.py }.
.4 { \includegraphics[width=0.05\textwidth]{tree/Actions-document-open-folder-icon} generator }.
.5 { \includegraphics[width=0.05\textwidth]{tree/Mimetypes-text-x-python-icon} fluid.py }.
.5 { \includegraphics[width=0.05\textwidth]{tree/Mimetypes-text-x-python-icon} vec.py }.
.5 { \includegraphics[width=0.05\textwidth]{tree/Mimetypes-text-x-python-icon} solid.py }.
.5 { \includegraphics[width=0.05\textwidth]{tree/Mimetypes-text-x-python-icon} \_\_init\_\_.py }.
.4 { \includegraphics[width=0.05\textwidth]{tree/Actions-document-open-folder-icon} mesh\_loader }.
.5 { \includegraphics[width=0.05\textwidth]{tree/Mimetypes-text-x-python-icon} Abaqus.py }.
.5 { \includegraphics[width=0.05\textwidth]{tree/Mimetypes-text-x-python-icon} GiD.py }.
.5 { \includegraphics[width=0.05\textwidth]{tree/Mimetypes-text-x-python-icon} \_\_init\_\_.py }.
.2 { \includegraphics[width=0.05\textwidth]{tree/Actions-document-open-folder-icon} examples }.
.3 { \includegraphics[width=0.05\textwidth]{tree/Mimetypes-text-plain-icon} CMakeLists.txt }.
.3 { \includegraphics[width=0.05\textwidth]{tree/Actions-document-open-folder-icon} LateralWater\_1x\_Ghost }.
.4 { \includegraphics[width=0.05\textwidth]{tree/Mimetypes-text-x-python-icon} Create.py }.
.4 { \includegraphics[width=0.05\textwidth]{tree/Actions-document-open-folder-icon} doc }.
.5 { \includegraphics[width=0.05\textwidth]{tree/Mimetypes-text-x-python-icon} plot.py }.
.5 { \includegraphics[width=0.05\textwidth]{tree/Mimetypes-application-pdf-icon} SOUTOIGLESIAS\_BOTIA\_SPHERIC\_TESTCASE\_10.pdf }.
.5 { \includegraphics[width=0.05\textwidth]{tree/Mimetypes-text-plain-icon} lateral\_water\_1x.txt }.
.4 { \includegraphics[width=0.05\textwidth]{tree/Actions-document-open-folder-icon} Move }.
.5 { \includegraphics[width=0.05\textwidth]{tree/Mimetypes-x-office-document-icon} 6DOFresources.ods }.
.5 { \includegraphics[width=0.05\textwidth]{tree/Mimetypes-text-plain-icon} 6DOF.dat }.
.3 { \includegraphics[width=0.05\textwidth]{tree/Actions-document-open-folder-icon} cMake }.
.4 { \includegraphics[width=0.05\textwidth]{tree/Actions-document-open-folder-icon} LateralWater\_1x\_Ghost }.
.5 { \includegraphics[width=0.05\textwidth]{tree/Mimetypes-application-xml-icon} Fluids.xml }.
.5 { \includegraphics[width=0.05\textwidth]{tree/Mimetypes-application-xml-icon} GhostParticles.xml }.
.5 { \includegraphics[width=0.05\textwidth]{tree/Mimetypes-application-xml-icon} Time.xml }.
.5 { \includegraphics[width=0.05\textwidth]{tree/Mimetypes-application-xml-icon} Sensors.xml }.
.5 { \includegraphics[width=0.05\textwidth]{tree/Mimetypes-application-x-shellscript-icon} run.sh }.
.5 { \includegraphics[width=0.05\textwidth]{tree/Mimetypes-application-xml-icon} Movements.xml }.
.5 { \includegraphics[width=0.05\textwidth]{tree/Mimetypes-application-xml-icon} Settings.xml }.
.5 { \includegraphics[width=0.05\textwidth]{tree/Mimetypes-application-xml-icon} SPH.xml }.
.5 { \includegraphics[width=0.05\textwidth]{tree/Mimetypes-application-xml-icon} Main.xml }.
.5 { \includegraphics[width=0.05\textwidth]{tree/Actions-document-open-folder-icon} Move }.
.6 { \includegraphics[width=0.05\textwidth]{tree/Mimetypes-application-xml-icon} 6DOF.xml }.
.4 { \includegraphics[width=0.05\textwidth]{tree/Actions-document-open-folder-icon} LateralWater\_1x\_Fix }.
.5 { \includegraphics[width=0.05\textwidth]{tree/Mimetypes-application-xml-icon} Fluids.xml }.
.5 { \includegraphics[width=0.05\textwidth]{tree/Mimetypes-application-xml-icon} Time.xml }.
.5 { \includegraphics[width=0.05\textwidth]{tree/Mimetypes-application-xml-icon} Sensors.xml }.
.5 { \includegraphics[width=0.05\textwidth]{tree/Mimetypes-application-x-shellscript-icon} run.sh }.
.5 { \includegraphics[width=0.05\textwidth]{tree/Mimetypes-application-xml-icon} Movements.xml }.
.5 { \includegraphics[width=0.05\textwidth]{tree/Mimetypes-application-xml-icon} Settings.xml }.
.5 { \includegraphics[width=0.05\textwidth]{tree/Mimetypes-application-xml-icon} SPH.xml }.
.5 { \includegraphics[width=0.05\textwidth]{tree/Mimetypes-application-xml-icon} Main.xml }.
.5 { \includegraphics[width=0.05\textwidth]{tree/Actions-document-open-folder-icon} Move }.
.6 { \includegraphics[width=0.05\textwidth]{tree/Mimetypes-application-xml-icon} 6DOF.xml }.
.4 { \includegraphics[width=0.05\textwidth]{tree/Actions-document-open-folder-icon} perezrojas\_etal\_stab\_2012 }.
.5 { \includegraphics[width=0.05\textwidth]{tree/Mimetypes-application-xml-icon} Fluids.xml }.
.5 { \includegraphics[width=0.05\textwidth]{tree/Mimetypes-application-xml-icon} Time.xml }.
.5 { \includegraphics[width=0.05\textwidth]{tree/Mimetypes-application-x-shellscript-icon} run.sh }.
.5 { \includegraphics[width=0.05\textwidth]{tree/Mimetypes-application-xml-icon} Movements.xml }.
.5 { \includegraphics[width=0.05\textwidth]{tree/Mimetypes-application-xml-icon} Settings.xml }.
.5 { \includegraphics[width=0.05\textwidth]{tree/Mimetypes-application-xml-icon} SPH.xml }.
.5 { \includegraphics[width=0.05\textwidth]{tree/Mimetypes-application-xml-icon} Main.xml }.
.5 { \includegraphics[width=0.05\textwidth]{tree/Actions-document-open-folder-icon} Move }.
.6 { \includegraphics[width=0.05\textwidth]{tree/Mimetypes-text-x-python-icon} move.py }.
.6 { \includegraphics[width=0.05\textwidth]{tree/Mimetypes-application-xml-icon} 6DOF.xml }.
.4 { \includegraphics[width=0.05\textwidth]{tree/Actions-document-open-folder-icon} lobovsky\_etal\_jfs\_2014 }.
.5 { \includegraphics[width=0.05\textwidth]{tree/Mimetypes-application-xml-icon} Fluids.xml }.
.5 { \includegraphics[width=0.05\textwidth]{tree/Mimetypes-application-xml-icon} Time.xml }.
.5 { \includegraphics[width=0.05\textwidth]{tree/Mimetypes-application-xml-icon} Sensors.xml }.
.5 { \includegraphics[width=0.05\textwidth]{tree/Mimetypes-application-x-shellscript-icon} run.sh }.
.5 { \includegraphics[width=0.05\textwidth]{tree/Mimetypes-application-xml-icon} Settings.xml }.
.5 { \includegraphics[width=0.05\textwidth]{tree/Mimetypes-application-xml-icon} SPH.xml }.
.5 { \includegraphics[width=0.05\textwidth]{tree/Mimetypes-application-xml-icon} Main.xml }.
.4 { \includegraphics[width=0.05\textwidth]{tree/Actions-document-open-folder-icon} LateralWater\_1x\_DeLeffe }.
.5 { \includegraphics[width=0.05\textwidth]{tree/Mimetypes-application-xml-icon} Fluids.xml }.
.5 { \includegraphics[width=0.05\textwidth]{tree/Mimetypes-application-xml-icon} Time.xml }.
.5 { \includegraphics[width=0.05\textwidth]{tree/Mimetypes-application-xml-icon} Sensors.xml }.
.5 { \includegraphics[width=0.05\textwidth]{tree/Mimetypes-application-x-shellscript-icon} run.sh }.
.5 { \includegraphics[width=0.05\textwidth]{tree/Mimetypes-application-xml-icon} Movements.xml }.
.5 { \includegraphics[width=0.05\textwidth]{tree/Mimetypes-application-xml-icon} Settings.xml }.
.5 { \includegraphics[width=0.05\textwidth]{tree/Mimetypes-application-xml-icon} SPH.xml }.
.5 { \includegraphics[width=0.05\textwidth]{tree/Mimetypes-application-xml-icon} Main.xml }.
.5 { \includegraphics[width=0.05\textwidth]{tree/Actions-document-open-folder-icon} Move }.
.6 { \includegraphics[width=0.05\textwidth]{tree/Mimetypes-application-xml-icon} 6DOF.xml }.
.3 { \includegraphics[width=0.05\textwidth]{tree/Actions-document-open-folder-icon} LateralWater\_1x\_Fix }.
.4 { \includegraphics[width=0.05\textwidth]{tree/Mimetypes-text-x-python-icon} Create.py }.
.4 { \includegraphics[width=0.05\textwidth]{tree/Actions-document-open-folder-icon} doc }.
.5 { \includegraphics[width=0.05\textwidth]{tree/Mimetypes-text-x-python-icon} plot.py }.
.5 { \includegraphics[width=0.05\textwidth]{tree/Mimetypes-application-pdf-icon} SOUTOIGLESIAS\_BOTIA\_SPHERIC\_TESTCASE\_10.pdf }.
.5 { \includegraphics[width=0.05\textwidth]{tree/Mimetypes-text-plain-icon} lateral\_water\_1x.txt }.
.4 { \includegraphics[width=0.05\textwidth]{tree/Actions-document-open-folder-icon} Move }.
.5 { \includegraphics[width=0.05\textwidth]{tree/Mimetypes-x-office-document-icon} 6DOFresources.ods }.
.5 { \includegraphics[width=0.05\textwidth]{tree/Mimetypes-text-plain-icon} 6DOF.dat }.
.3 { \includegraphics[width=0.05\textwidth]{tree/Actions-document-open-folder-icon} perezrojas\_etal\_stab\_2012 }.
.4 { \includegraphics[width=0.05\textwidth]{tree/Mimetypes-text-x-python-icon} Create.py }.
.4 { \includegraphics[width=0.05\textwidth]{tree/Actions-document-open-folder-icon} doc }.
.5 { \includegraphics[width=0.05\textwidth]{tree/Mimetypes-text-x-python-icon} plot.py }.
.5 { \includegraphics[width=0.05\textwidth]{tree/Mimetypes-application-pdf-icon} SOUTOIGLESIAS\_BOTIA\_SPHERIC\_TESTCASE9\_TLD.pdf }.
.4 { \includegraphics[width=0.05\textwidth]{tree/Actions-document-open-folder-icon} Move }.
.5 { \includegraphics[width=0.05\textwidth]{tree/Mimetypes-text-plain-icon} T\_1-94\_A100mm\_water.dat }.
.3 { \includegraphics[width=0.05\textwidth]{tree/Actions-document-open-folder-icon} lobovsky\_etal\_jfs\_2014 }.
.4 { \includegraphics[width=0.05\textwidth]{tree/Mimetypes-text-x-python-icon} Create.py }.
.4 { \includegraphics[width=0.05\textwidth]{tree/Actions-document-open-folder-icon} doc }.
.5 { \includegraphics[width=0.05\textwidth]{tree/Mimetypes-text-plain-icon} Fig30\_filtered\_5.dat }.
.5 { \includegraphics[width=0.05\textwidth]{tree/Mimetypes-text-plain-icon} Fig20\_filtered\_2.dat }.
.5 { \includegraphics[width=0.05\textwidth]{tree/Mimetypes-text-plain-icon} Fig12\_02\_ETSIN\_Front\_Hu\_Koshizuka\_3.dat }.
.5 { \includegraphics[width=0.05\textwidth]{tree/Mimetypes-text-plain-icon} Fig29\_filtered\_3.dat }.
.5 { \includegraphics[width=0.05\textwidth]{tree/Mimetypes-text-plain-icon} Fig31\_filtered\_1.dat }.
.5 { \includegraphics[width=0.05\textwidth]{tree/Mimetypes-text-plain-icon} Fig12\_01\_ETSIN\_Front\_Dressler\_Martin\_4.dat }.
.5 { \includegraphics[width=0.05\textwidth]{tree/Mimetypes-text-x-python-icon} sensor3.py }.
.5 { \includegraphics[width=0.05\textwidth]{tree/Mimetypes-text-plain-icon} Fig12\_01\_ETSIN\_Front\_Dressler\_Martin\_7.dat }.
.5 { \includegraphics[width=0.05\textwidth]{tree/Mimetypes-text-plain-icon} Fig12\_01\_ETSIN\_Front\_Dressler\_Martin\_6.dat }.
.5 { \includegraphics[width=0.05\textwidth]{tree/Mimetypes-text-plain-icon} Fig30\_filtered\_2.dat }.
.5 { \includegraphics[width=0.05\textwidth]{tree/Mimetypes-text-plain-icon} Fig12\_01\_ETSIN\_Front\_Dressler\_Martin\_3.dat }.
.5 { \includegraphics[width=0.05\textwidth]{tree/Mimetypes-text-plain-icon} Fig29\_filtered\_2.dat }.
.5 { \includegraphics[width=0.05\textwidth]{tree/Mimetypes-text-plain-icon} Fig31\_filtered\_4.dat }.
.5 { \includegraphics[width=0.05\textwidth]{tree/Mimetypes-text-plain-icon} Fig12\_01\_ETSIN\_Front\_Dressler\_Martin\_5.dat }.
.5 { \includegraphics[width=0.05\textwidth]{tree/Mimetypes-text-plain-icon} Fig31\_filtered\_2.dat }.
.5 { \includegraphics[width=0.05\textwidth]{tree/Mimetypes-text-plain-icon} Fig30\_filtered\_4.dat }.
.5 { \includegraphics[width=0.05\textwidth]{tree/Mimetypes-text-x-python-icon} performance.py }.
.5 { \includegraphics[width=0.05\textwidth]{tree/Mimetypes-text-plain-icon} Fig12\_01\_ETSIN\_Front\_Dressler\_Martin\_2.dat }.
.5 { \includegraphics[width=0.05\textwidth]{tree/Mimetypes-text-plain-icon} Fig29\_filtered\_4.dat }.
.5 { \includegraphics[width=0.05\textwidth]{tree/Mimetypes-text-plain-icon} Fig12\_02\_ETSIN\_Front\_Hu\_Koshizuka\_5.dat }.
.5 { \includegraphics[width=0.05\textwidth]{tree/Mimetypes-text-x-python-icon} sensor2.py }.
.5 { \includegraphics[width=0.05\textwidth]{tree/Mimetypes-text-x-python-icon} sensor1.py }.
.5 { \includegraphics[width=0.05\textwidth]{tree/Mimetypes-text-x-python-icon} sensor4.py }.
.5 { \includegraphics[width=0.05\textwidth]{tree/Mimetypes-text-plain-icon} Fig30\_filtered\_1.dat }.
.5 { \includegraphics[width=0.05\textwidth]{tree/Mimetypes-text-plain-icon} Fig29\_filtered\_1.dat }.
.5 { \includegraphics[width=0.05\textwidth]{tree/Mimetypes-text-plain-icon} Fig12\_02\_ETSIN\_Front\_Hu\_Koshizuka\_4.dat }.
.5 { \includegraphics[width=0.05\textwidth]{tree/Mimetypes-text-x-python-icon} sensors.py }.
.5 { \includegraphics[width=0.05\textwidth]{tree/Mimetypes-text-plain-icon} Fig20\_filtered\_3.dat }.
.5 { \includegraphics[width=0.05\textwidth]{tree/Mimetypes-text-plain-icon} Fig30\_filtered\_3.dat }.
.5 { \includegraphics[width=0.05\textwidth]{tree/Mimetypes-text-plain-icon} Fig20\_filtered\_1.dat }.
.5 { \includegraphics[width=0.05\textwidth]{tree/Mimetypes-text-plain-icon} Fig31\_filtered\_3.dat }.
.3 { \includegraphics[width=0.05\textwidth]{tree/Actions-document-open-folder-icon} LateralWater\_1x\_DeLeffe }.
.4 { \includegraphics[width=0.05\textwidth]{tree/Mimetypes-text-x-python-icon} Create.py }.
.4 { \includegraphics[width=0.05\textwidth]{tree/Actions-document-open-folder-icon} doc }.
.5 { \includegraphics[width=0.05\textwidth]{tree/Mimetypes-text-x-python-icon} plot.py }.
.5 { \includegraphics[width=0.05\textwidth]{tree/Mimetypes-application-pdf-icon} SOUTOIGLESIAS\_BOTIA\_SPHERIC\_TESTCASE\_10.pdf }.
.5 { \includegraphics[width=0.05\textwidth]{tree/Mimetypes-text-plain-icon} lateral\_water\_1x.txt }.
.4 { \includegraphics[width=0.05\textwidth]{tree/Actions-document-open-folder-icon} Move }.
.5 { \includegraphics[width=0.05\textwidth]{tree/Mimetypes-x-office-document-icon} 6DOFresources.ods }.
.5 { \includegraphics[width=0.05\textwidth]{tree/Mimetypes-text-plain-icon} 6DOF.dat }.
.2 { \includegraphics[width=0.05\textwidth]{tree/Actions-document-open-folder-icon} cMake }.
.3 { \includegraphics[width=0.05\textwidth]{tree/Mimetypes-text-plain-icon} FindOpenCL.cmake }.
.3 { \includegraphics[width=0.05\textwidth]{tree/Mimetypes-text-plain-icon} Findhdf5.cmake }.
.3 { \includegraphics[width=0.05\textwidth]{tree/Mimetypes-text-plain-icon} FindCurses.cmake }.
.3 { \includegraphics[width=0.05\textwidth]{tree/Mimetypes-text-plain-icon} FindXerces.cmake }.
.3 { \includegraphics[width=0.05\textwidth]{tree/Mimetypes-text-plain-icon} ConfigureChecks.cmake }.
.3 { \includegraphics[width=0.05\textwidth]{tree/Mimetypes-text-plain-icon} FindH5Part.cmake }.
.3 { \includegraphics[width=0.05\textwidth]{tree/Mimetypes-text-plain-icon} FindEigen3.cmake }.
.3 { \includegraphics[width=0.05\textwidth]{tree/Mimetypes-text-plain-icon} Findmatheval.cmake }.
.2 { \includegraphics[width=0.05\textwidth]{tree/Actions-document-open-folder-icon} include }.
.3 { \includegraphics[width=0.05\textwidth]{tree/Mimetypes-text-plain-icon} CMakeLists.txt }.
.3 { \includegraphics[width=0.05\textwidth]{tree/Mimetypes-text-x-chdr-icon} ArgumentsManager.h }.
.3 { \includegraphics[width=0.05\textwidth]{tree/Mimetypes-text-x-chdr-icon} sphPrerequisites.h }.
.3 { \includegraphics[width=0.05\textwidth]{tree/Mimetypes-text-x-chdr-icon} Singleton.h }.
.3 { \includegraphics[width=0.05\textwidth]{tree/Mimetypes-text-x-chdr-icon} ScreenManager.h }.
.3 { \includegraphics[width=0.05\textwidth]{tree/Mimetypes-text-x-chdr-icon} AuxiliarMethods.h }.
.3 { \includegraphics[width=0.05\textwidth]{tree/Mimetypes-text-x-chdr-icon} ProblemSetup.h }.
.3 { \includegraphics[width=0.05\textwidth]{tree/Mimetypes-text-x-chdr-icon} TimeManager.h }.
.3 { \includegraphics[width=0.05\textwidth]{tree/Mimetypes-text-x-chdr-icon} Fluid.h }.
.3 { \includegraphics[width=0.05\textwidth]{tree/Mimetypes-text-x-chdr-icon} CalcServer.h }.
.3 { \includegraphics[width=0.05\textwidth]{tree/Mimetypes-text-x-chdr-icon} FileManager.h }.
.3 { \includegraphics[width=0.05\textwidth]{tree/Actions-document-open-folder-icon} Tokenizer }.
.4 { \includegraphics[width=0.05\textwidth]{tree/Mimetypes-text-x-chdr-icon} Tokenizer.h }.
.3 { \includegraphics[width=0.05\textwidth]{tree/Actions-document-open-folder-icon} CalcServer }.
.4 { \includegraphics[width=0.05\textwidth]{tree/Mimetypes-text-x-chdr-icon} Torque.h }.
.4 { \includegraphics[width=0.05\textwidth]{tree/Mimetypes-text-x-chdr-icon} RadixSort.h }.
.4 { \includegraphics[width=0.05\textwidth]{tree/Mimetypes-text-x-chdr-icon} Domain.h }.
.4 { \includegraphics[width=0.05\textwidth]{tree/Mimetypes-text-x-chdr-icon} Sensors.h }.
.4 { \includegraphics[width=0.05\textwidth]{tree/Mimetypes-text-x-chdr-icon} Bounds.h }.
.4 { \includegraphics[width=0.05\textwidth]{tree/Mimetypes-text-x-chdr-icon} Corrector.h }.
.4 { \includegraphics[width=0.05\textwidth]{tree/Mimetypes-text-x-chdr-icon} LinkList.h }.
.4 { \includegraphics[width=0.05\textwidth]{tree/Mimetypes-text-x-chdr-icon} Permutate.h }.
.4 { \includegraphics[width=0.05\textwidth]{tree/Mimetypes-text-x-chdr-icon} Grid.h }.
.4 { \includegraphics[width=0.05\textwidth]{tree/Mimetypes-text-x-chdr-icon} TimeStep.h }.
.4 { \includegraphics[width=0.05\textwidth]{tree/Mimetypes-text-x-chdr-icon} Rates.h }.
.4 { \includegraphics[width=0.05\textwidth]{tree/Mimetypes-text-x-chdr-icon} DensityInterpolation.h }.
.4 { \includegraphics[width=0.05\textwidth]{tree/Mimetypes-text-x-chdr-icon} Energy.h }.
.4 { \includegraphics[width=0.05\textwidth]{tree/Mimetypes-text-x-chdr-icon} Reduction.h }.
.4 { \includegraphics[width=0.05\textwidth]{tree/Mimetypes-text-x-chdr-icon} Kernel.h }.
.4 { \includegraphics[width=0.05\textwidth]{tree/Mimetypes-text-x-chdr-icon} Shepard.h }.
.4 { \includegraphics[width=0.05\textwidth]{tree/Mimetypes-text-x-chdr-icon} Predictor.h }.
.4 { \includegraphics[width=0.05\textwidth]{tree/Actions-document-open-folder-icon} Movements }.
.5 { \includegraphics[width=0.05\textwidth]{tree/Mimetypes-text-x-chdr-icon} Quaternion.h }.
.5 { \includegraphics[width=0.05\textwidth]{tree/Mimetypes-text-x-chdr-icon} ScriptQuaternion.h }.
.5 { \includegraphics[width=0.05\textwidth]{tree/Mimetypes-text-x-chdr-icon} C1Quaternion.h }.
.5 { \includegraphics[width=0.05\textwidth]{tree/Mimetypes-text-x-chdr-icon} Movement.h }.
.5 { \includegraphics[width=0.05\textwidth]{tree/Mimetypes-text-x-chdr-icon} LinearInterpolation.h }.
.5 { \includegraphics[width=0.05\textwidth]{tree/Mimetypes-text-x-chdr-icon} C1Interpolation.h }.
.5 { \includegraphics[width=0.05\textwidth]{tree/Mimetypes-text-x-chdr-icon} LIQuaternion.h }.
.4 { \includegraphics[width=0.05\textwidth]{tree/Actions-document-open-folder-icon} Portal }.
.5 { \includegraphics[width=0.05\textwidth]{tree/Mimetypes-text-x-chdr-icon} Portal.h }.
.4 { \includegraphics[width=0.05\textwidth]{tree/Actions-document-open-folder-icon} Boundary }.
.5 { \includegraphics[width=0.05\textwidth]{tree/Mimetypes-text-x-chdr-icon} ElasticBounce.h }.
.5 { \includegraphics[width=0.05\textwidth]{tree/Mimetypes-text-x-chdr-icon} GhostParticles.h }.
.5 { \includegraphics[width=0.05\textwidth]{tree/Mimetypes-text-x-chdr-icon} DeLeffe.h }.
.3 { \includegraphics[width=0.05\textwidth]{tree/Actions-document-open-folder-icon} InputOutput }.
.4 { \includegraphics[width=0.05\textwidth]{tree/Mimetypes-text-x-chdr-icon} Log.h }.
.4 { \includegraphics[width=0.05\textwidth]{tree/Mimetypes-text-x-chdr-icon} ASCII.h }.
.4 { \includegraphics[width=0.05\textwidth]{tree/Mimetypes-text-x-chdr-icon} Report.h }.
.4 { \includegraphics[width=0.05\textwidth]{tree/Mimetypes-text-x-chdr-icon} InputOutput.h }.
.4 { \includegraphics[width=0.05\textwidth]{tree/Mimetypes-text-x-chdr-icon} State.h }.
.4 { \includegraphics[width=0.05\textwidth]{tree/Mimetypes-text-x-chdr-icon} Bounds.h }.
.4 { \includegraphics[width=0.05\textwidth]{tree/Mimetypes-text-x-chdr-icon} VTK.h }.
.4 { \includegraphics[width=0.05\textwidth]{tree/Mimetypes-text-x-chdr-icon} Energy.h }.
.4 { \includegraphics[width=0.05\textwidth]{tree/Mimetypes-text-x-chdr-icon} Particles.h }.
.2 { \includegraphics[width=0.05\textwidth]{tree/Actions-document-open-folder-icon} resources }.
.3 { \includegraphics[width=0.05\textwidth]{tree/Mimetypes-x-office-document-icon} OpenCLMain.xml.in }.
.3 { \includegraphics[width=0.05\textwidth]{tree/Mimetypes-text-plain-icon} CMakeLists.txt }.
.3 { \includegraphics[width=0.05\textwidth]{tree/Actions-document-open-folder-icon} OpenCL }.
.4 { \includegraphics[width=0.05\textwidth]{tree/Mimetypes-text-x-csrc-icon} LinkList.cl }.
.4 { \includegraphics[width=0.05\textwidth]{tree/Mimetypes-text-x-csrc-icon} Corrector.cl }.
.4 { \includegraphics[width=0.05\textwidth]{tree/Mimetypes-text-x-chdr-icon} RatesSensors.hcl }.
.4 { \includegraphics[width=0.05\textwidth]{tree/Mimetypes-text-x-csrc-icon} Shepard.cl }.
.4 { \includegraphics[width=0.05\textwidth]{tree/Mimetypes-text-x-csrc-icon} Rates.cl }.
.4 { \includegraphics[width=0.05\textwidth]{tree/Mimetypes-text-x-chdr-icon} RatesBounds.hcl }.
.4 { \includegraphics[width=0.05\textwidth]{tree/Mimetypes-text-x-csrc-icon} Energy.cl }.
.4 { \includegraphics[width=0.05\textwidth]{tree/Mimetypes-text-x-csrc-icon} Permutate.cl }.
.4 { \includegraphics[width=0.05\textwidth]{tree/Mimetypes-text-x-csrc-icon} DensInt.cl }.
.4 { \includegraphics[width=0.05\textwidth]{tree/Mimetypes-text-x-csrc-icon} RadixSort.cl }.
.4 { \includegraphics[width=0.05\textwidth]{tree/Mimetypes-text-x-csrc-icon} Torque.cl }.
.4 { \includegraphics[width=0.05\textwidth]{tree/Mimetypes-text-x-csrc-icon} Bounds.cl }.
.4 { \includegraphics[width=0.05\textwidth]{tree/Mimetypes-text-x-chdr-icon} DensInt.hcl }.
.4 { \includegraphics[width=0.05\textwidth]{tree/Mimetypes-text-x-chdr-icon} Rates.hcl }.
.4 { \includegraphics[width=0.05\textwidth]{tree/Mimetypes-text-x-csrc-icon} Predictor.cl }.
.4 { \includegraphics[width=0.05\textwidth]{tree/Mimetypes-text-x-csrc-icon} Sensors.cl }.
.4 { \includegraphics[width=0.05\textwidth]{tree/Mimetypes-text-x-csrc-icon} TimeStep.cl }.
.4 { \includegraphics[width=0.05\textwidth]{tree/Mimetypes-text-x-csrc-icon} Domain.cl }.
.4 { \includegraphics[width=0.05\textwidth]{tree/Mimetypes-text-x-csrc-icon} Reduction.cl }.
.4 { \includegraphics[width=0.05\textwidth]{tree/Actions-document-open-folder-icon} Movements }.
.5 { \includegraphics[width=0.05\textwidth]{tree/Mimetypes-text-x-csrc-icon} Quaternion.cl }.
.4 { \includegraphics[width=0.05\textwidth]{tree/Actions-document-open-folder-icon} KernelFunctions }.
.5 { \includegraphics[width=0.05\textwidth]{tree/Mimetypes-text-x-chdr-icon} Wendland3D.hcl }.
.5 { \includegraphics[width=0.05\textwidth]{tree/Mimetypes-text-x-chdr-icon} Gaussian2D.hcl }.
.5 { \includegraphics[width=0.05\textwidth]{tree/Mimetypes-text-x-chdr-icon} Gaussian3D.hcl }.
.5 { \includegraphics[width=0.05\textwidth]{tree/Mimetypes-text-x-chdr-icon} CubicSpline2D.hcl }.
.5 { \includegraphics[width=0.05\textwidth]{tree/Mimetypes-text-x-chdr-icon} CubicSpline3D.hcl }.
.5 { \includegraphics[width=0.05\textwidth]{tree/Mimetypes-text-x-chdr-icon} Wendland2D.hcl }.
.4 { \includegraphics[width=0.05\textwidth]{tree/Actions-document-open-folder-icon} Portal }.
.5 { \includegraphics[width=0.05\textwidth]{tree/Mimetypes-text-x-csrc-icon} Portal.cl }.
.4 { \includegraphics[width=0.05\textwidth]{tree/Actions-document-open-folder-icon} Boundary }.
.5 { \includegraphics[width=0.05\textwidth]{tree/Mimetypes-text-x-chdr-icon} GhostParticles.hcl }.
.5 { \includegraphics[width=0.05\textwidth]{tree/Mimetypes-text-x-chdr-icon} Wall.hcl }.
.5 { \includegraphics[width=0.05\textwidth]{tree/Mimetypes-text-x-chdr-icon} DeLeffe.hcl }.
.5 { \includegraphics[width=0.05\textwidth]{tree/Mimetypes-text-x-csrc-icon} GhostParticles.cl }.
.5 { \includegraphics[width=0.05\textwidth]{tree/Mimetypes-text-x-chdr-icon} ElasticBounce.hcl }.
.5 { \includegraphics[width=0.05\textwidth]{tree/Mimetypes-text-x-csrc-icon} DeLeffe.cl }.
.5 { \includegraphics[width=0.05\textwidth]{tree/Mimetypes-text-x-csrc-icon} ElasticBounce.cl }.
.4 { \includegraphics[width=0.05\textwidth]{tree/Actions-document-open-folder-icon} types }.
.5 { \includegraphics[width=0.05\textwidth]{tree/Mimetypes-text-x-chdr-icon} 3D.h }.
.5 { \includegraphics[width=0.05\textwidth]{tree/Mimetypes-text-x-chdr-icon} 2D.h }.
.2 { \includegraphics[width=0.05\textwidth]{tree/Actions-document-open-folder-icon} CodeBlocks }.
.3 { \includegraphics[width=0.05\textwidth]{tree/Mimetypes-x-office-document-icon} AQUAgpusph.cbp }.
.3 { \includegraphics[width=0.05\textwidth]{tree/Mimetypes-x-office-document-icon} AQUAgpusph.layout }.
.2 { \includegraphics[width=0.05\textwidth]{tree/Actions-document-open-folder-icon} src }.
.3 { \includegraphics[width=0.05\textwidth]{tree/Mimetypes-text-x-c-plus-plus-src-icon} TimeManager.cpp }.
.3 { \includegraphics[width=0.05\textwidth]{tree/Mimetypes-text-x-c-plus-plus-src-icon} FileManager.cpp }.
.3 { \includegraphics[width=0.05\textwidth]{tree/Mimetypes-text-x-c-plus-plus-src-icon} AuxiliarMethods.cpp }.
.3 { \includegraphics[width=0.05\textwidth]{tree/Mimetypes-text-plain-icon} CMakeLists.txt }.
.3 { \includegraphics[width=0.05\textwidth]{tree/Mimetypes-text-x-c-plus-plus-src-icon} ProblemSetup.cpp }.
.3 { \includegraphics[width=0.05\textwidth]{tree/Mimetypes-text-x-c-plus-plus-src-icon} Fluid.cpp }.
.3 { \includegraphics[width=0.05\textwidth]{tree/Mimetypes-text-x-c-plus-plus-src-icon} ArgumentsManager.cpp }.
.3 { \includegraphics[width=0.05\textwidth]{tree/Mimetypes-text-x-c-plus-plus-src-icon} main.cpp }.
.3 { \includegraphics[width=0.05\textwidth]{tree/Mimetypes-text-x-c-plus-plus-src-icon} ScreenManager.cpp }.
.3 { \includegraphics[width=0.05\textwidth]{tree/Actions-document-open-folder-icon} Tokenizer }.
.4 { \includegraphics[width=0.05\textwidth]{tree/Mimetypes-text-x-c-plus-plus-src-icon} Tokenizer.cpp }.
.3 { \includegraphics[width=0.05\textwidth]{tree/Actions-document-open-folder-icon} CalcServer }.
.4 { \includegraphics[width=0.05\textwidth]{tree/Mimetypes-text-x-c-plus-plus-src-icon} Rates.cpp }.
.4 { \includegraphics[width=0.05\textwidth]{tree/Mimetypes-text-x-c-plus-plus-src-icon} DensityInterpolation.cpp }.
.4 { \includegraphics[width=0.05\textwidth]{tree/Mimetypes-text-x-c-plus-plus-src-icon} Predictor.cpp }.
.4 { \includegraphics[width=0.05\textwidth]{tree/Mimetypes-text-x-c-plus-plus-src-icon} Grid.cpp }.
.4 { \includegraphics[width=0.05\textwidth]{tree/Mimetypes-text-x-c-plus-plus-src-icon} Energy.cpp }.
.4 { \includegraphics[width=0.05\textwidth]{tree/Mimetypes-text-x-c-plus-plus-src-icon} Domain.cpp }.
.4 { \includegraphics[width=0.05\textwidth]{tree/Mimetypes-text-x-c-plus-plus-src-icon} Corrector.cpp }.
.4 { \includegraphics[width=0.05\textwidth]{tree/Mimetypes-text-x-c-plus-plus-src-icon} CalcServer.cpp }.
.4 { \includegraphics[width=0.05\textwidth]{tree/Mimetypes-text-x-c-plus-plus-src-icon} TimeStep.cpp }.
.4 { \includegraphics[width=0.05\textwidth]{tree/Mimetypes-text-plain-icon} CMakeLists.txt }.
.4 { \includegraphics[width=0.05\textwidth]{tree/Mimetypes-text-x-c-plus-plus-src-icon} LinkList.cpp }.
.4 { \includegraphics[width=0.05\textwidth]{tree/Mimetypes-text-x-c-plus-plus-src-icon} Sensors.cpp }.
.4 { \includegraphics[width=0.05\textwidth]{tree/Mimetypes-text-x-c-plus-plus-src-icon} Torque.cpp }.
.4 { \includegraphics[width=0.05\textwidth]{tree/Mimetypes-text-x-c-plus-plus-src-icon} Permutate.cpp }.
.4 { \includegraphics[width=0.05\textwidth]{tree/Mimetypes-text-x-c-plus-plus-src-icon} Reduction.cpp }.
.4 { \includegraphics[width=0.05\textwidth]{tree/Mimetypes-text-x-c-plus-plus-src-icon} RadixSort.cpp }.
.4 { \includegraphics[width=0.05\textwidth]{tree/Mimetypes-text-x-c-plus-plus-src-icon} Kernel.cpp }.
.4 { \includegraphics[width=0.05\textwidth]{tree/Mimetypes-text-x-c-plus-plus-src-icon} Shepard.cpp }.
.4 { \includegraphics[width=0.05\textwidth]{tree/Mimetypes-text-x-c-plus-plus-src-icon} Bounds.cpp }.
.4 { \includegraphics[width=0.05\textwidth]{tree/Actions-document-open-folder-icon} Movements }.
.5 { \includegraphics[width=0.05\textwidth]{tree/Mimetypes-text-x-c-plus-plus-src-icon} C1Interpolation.cpp }.
.5 { \includegraphics[width=0.05\textwidth]{tree/Mimetypes-text-x-c-plus-plus-src-icon} LinearInterpolation.cpp }.
.5 { \includegraphics[width=0.05\textwidth]{tree/Mimetypes-text-x-c-plus-plus-src-icon} C1Quaternion.cpp }.
.5 { \includegraphics[width=0.05\textwidth]{tree/Mimetypes-text-x-c-plus-plus-src-icon} ScriptQuaternion.cpp }.
.5 { \includegraphics[width=0.05\textwidth]{tree/Mimetypes-text-x-c-plus-plus-src-icon} LIQuaternion.cpp }.
.5 { \includegraphics[width=0.05\textwidth]{tree/Mimetypes-text-x-c-plus-plus-src-icon} Quaternion.cpp }.
.5 { \includegraphics[width=0.05\textwidth]{tree/Mimetypes-text-x-c-plus-plus-src-icon} Movement.cpp }.
.4 { \includegraphics[width=0.05\textwidth]{tree/Actions-document-open-folder-icon} Portal }.
.5 { \includegraphics[width=0.05\textwidth]{tree/Mimetypes-text-x-c-plus-plus-src-icon} Portal.cpp }.
.4 { \includegraphics[width=0.05\textwidth]{tree/Actions-document-open-folder-icon} Boundary }.
.5 { \includegraphics[width=0.05\textwidth]{tree/Mimetypes-text-x-c-plus-plus-src-icon} DeLeffe.cpp }.
.5 { \includegraphics[width=0.05\textwidth]{tree/Mimetypes-text-x-c-plus-plus-src-icon} GhostParticles.cpp }.
.5 { \includegraphics[width=0.05\textwidth]{tree/Mimetypes-text-x-c-plus-plus-src-icon} ElasticBounce.cpp }.
.3 { \includegraphics[width=0.05\textwidth]{tree/Actions-document-open-folder-icon} InputOutput }.
.4 { \includegraphics[width=0.05\textwidth]{tree/Mimetypes-text-x-c-plus-plus-src-icon} Log.cpp }.
.4 { \includegraphics[width=0.05\textwidth]{tree/Mimetypes-text-x-c-plus-plus-src-icon} Energy.cpp }.
.4 { \includegraphics[width=0.05\textwidth]{tree/Mimetypes-text-x-c-plus-plus-src-icon} VTK.cpp }.
.4 { \includegraphics[width=0.05\textwidth]{tree/Mimetypes-text-x-c-plus-plus-src-icon} Particles.cpp }.
.4 { \includegraphics[width=0.05\textwidth]{tree/Mimetypes-text-x-c-plus-plus-src-icon} State.cpp }.
.4 { \includegraphics[width=0.05\textwidth]{tree/Mimetypes-text-x-c-plus-plus-src-icon} ASCII.cpp }.
.4 { \includegraphics[width=0.05\textwidth]{tree/Mimetypes-text-x-c-plus-plus-src-icon} Report.cpp }.
.4 { \includegraphics[width=0.05\textwidth]{tree/Mimetypes-text-x-c-plus-plus-src-icon} Bounds.cpp }.
}


\vspace{0.5cm}
\rule{\textwidth}{1pt}
\vspace{0.5cm}

All the files called ``\texttt{CMakeLists.txt}'', ``\texttt{*.cmake}'', or 
included in a ``cMake'' folder are designed to let cmake to configure the 
package for the build and installation process, please refer to section 
\ref{s:install} to learn more about cmake.
%
More specifically, the files called ``\texttt{CMakeLists.txt}'' are the ones 
where the instructions for cmake are written, while the other ones are 
templates that cmake is conveniently reading and editing.

In the folder ``\texttt{doc}'' the main documentation files can be found.
%
Inside can be found the subfolder ``\texttt{userManual}'', where this manual 
and all the resources to build it are placed.
%
During the cmake configuration, if the flag \textbf{AQUAGPUSPH\_BUILD\_DOC} 
is  activated (see section \ref{sss:install:cmake}), Doxygen documentation 
will be  build as well, placing it in the folder ``\texttt{doc/Doxygen/html}'' 
(see section \ref{ss:aquagpusph:general}).

In the folders ``\texttt{src}'' and ``\texttt{include}'' the source codes of 
\NAME can be found.
%
Note that the OpenCL codes, which are not included into the \NAME compiled 
binary file, are placed inside ``\texttt{resources}'' folder.
%
A CodeBlocks project is provided for the developers in the file 
``CodeBlocks/AQUAgpusph.cbp''.
%
However, it is strongly recommended to use cmake to perform compilations.

In the folder ``\texttt{examples}'' the examples provided with \NAME package 
are placed.
%
To setup the examples you should activate the flag 
\textbf{AQUAGPUSPH\_BUILD\_EXAMPLES}.
%
The examples available depends on whether the 2D version or 3D version is 
built (see section \ref{sss:install:cmake}).
%
When 2D version is build (\textbf{AQUAGPUSPH\_3D=OFF}) the following examples 
will be generated:
%
\begin{itemize}
	\item ``\texttt{examples/LateralWater\_1x\_DeLeffe}''
	\item ``\texttt{examples/LateralWater\_1x\_Fix}''
	\item ``\texttt{examples/LateralWater\_1x\_Ghost}''
	\item ``\texttt{examples/lobovsky\_etal\_jfs\_2014}''
\end{itemize}
%
While for the 3D version (\textbf{AQUAGPUSPH\_3D=ON}) the following examples 
will be generated:
%
\begin{itemize}
	\item ``\texttt{examples/perezrojas\_etal\_stab\_2012}''
\end{itemize}

Note that both, 3D and 2D versions, may be installed together in the same 
system.
%
After building the examples a bash script ``\texttt{run.sh}'' is provided for 
each one.
%
Call this script with the option ``\texttt{--help}'' to know how to run and 
track them.