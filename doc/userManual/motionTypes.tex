\subsection{Motion types}
\label{sss:aquagpusph:motions:Types}
%
\begin{center}
\textbf{Quaternion based motions}
\end{center}
%
The main method to impose motions in \NAME is to use Quaternions, that have the key feature that the motion is
ever fully described, not depending on any criteria. The problem with Quaternion imposed motions is that will
overwrite all the previous motions, so if you want to superpose several motions, Quaternion one must be the first.\\
%
When Quaternion motion is imposed in reality you are setting a movable reference coordinates, where object is fixed,
So in order to define instantaneous quaternion you must provide enough data to can describe it.
%
\begin{enumerate}
	\item 2D Simulations (AQUAgpusph2D)
	\begin{enumerate}
		\item \textbf{C.x}: x coordinate of the quaternion center.
		\item \textbf{C.y}: y coordinate of the quaternion center.
		\item \textbf{X.x}: x component of the X quaternion axis.
		\item \textbf{X.y}: y component of the X quaternion axis.
	\end{enumerate}
	\item 3D Simulations (AQUAgpusph)
	\begin{enumerate}
		\item \textbf{C.x}: x coordinate of the quaternion center.
		\item \textbf{C.y}: y coordinate of the quaternion center.
		\item \textbf{C.z}: z coordinate of the quaternion center.
		\item \textbf{X.x}: x component of the X quaternion axis.
		\item \textbf{X.y}: y component of the X quaternion axis.
		\item \textbf{X.z}: z component of the X quaternion axis.
		\item \textbf{Y.x}: x component of the Y quaternion axis.
		\item \textbf{Y.y}: y component of the Y quaternion axis.
		\item \textbf{Y.z}: z component of the Y quaternion axis.
	\end{enumerate}
\end{enumerate}
%
The undefined axis will be computed considering an orthonormal base. Provided axes will not be normalized, so
please ensure to provide right axes data.\\
%
Quaternion motions provided in \NAME will apply the motion over the particles with $imove \le 0$ flag (Fixed
particles or vertexes) and over defined walls as described on section \ref{sss:XML:GhostParticles}. You can
customize affected motion particles editing the provided Quaternion OpenCL kernel, but not walls for `Ghost
particles' boundary condition, that requires editing the \NAME source code.\\
%
In almost SPH simulations the time step is really small, for this reason in \NAME the velocity of the boundary
points (fixed particles, vertexes, or wall points) is determined as an euler derivation
%
\[v_{t+dt} = \frac{x_{t+dt} - x_t}{dt}\]
%