\chapter{Case setup options glossary}
\label{s:caseSetup}
%
\section{General}
%
\NAME has a powerful XML files input interface used to setup simulation cases.
Those files can be easily structured as a tree with the including capabilities.
XML is a standard file format where a schematic structure of the files is
allowed. You can learn more about XML syntax and possibilities at next web page:
%
\begin{center}
    \url{http://www.w3.org/XML}
\end{center}
%
The objective of this chapter is to show all the available settings and switches
that can be established for a \NAME simulation, but for a practical approach
to generating and running simulations please refer to chapter \ref{s:examples},
where \NAME package examples are discussed.
%
\section{XML definition}
%
\subsection{General}
%
Each XML file must have the following structure:
%
\begin{verbatim}
<?xml version="1.0" ?>
<sphInput>
    ...
</sphInput>
\end{verbatim}
%
Of course all the XML valid syntax can be used in these files
(i.e.- Comments, ...). In the table \ref{tables:caseSetup:XML:Options} all the
valid options that can be set at sphInput group are described. In the table
\ref{tables:caseSetup:XML:Groups} the valid sections into sphInput group are
described. You can add several several times each section which, depending on
the settings affected, will overwrite or add the information.
%
The different group tags are now discussed.
%
\begin{table}[h!b!p!]\small
	\centering
	\begin{tabular}{| c | c | c | l | }
		\hline
		\cellcolor[rgb]{0.7,0.7,0.7}Tag & \cellcolor[rgb]{0.7,0.7,0.7}Attributes & \cellcolor[rgb]{0.7,0.7,0.7}Valid values & \cellcolor[rgb]{0.7,0.7,0.7}Description \\
		\hline
		Include & file & XML file path & Specified XML file will processed before continuing parsing this file. \\
		        &      &               & You can include as many files as you want. \\
		\hline
	\end{tabular}
	\caption{Valid option tags in the sphInput group.}
	\label{tables:caseSetup:XML:Options}
\end{table}
%
\begin{table}[h!b!p!]\small
	\centering
	\begin{tabular}{| c | c | c | l | }
		\hline
		\cellcolor[rgb]{0.7,0.7,0.7}Tag & \cellcolor[rgb]{0.7,0.7,0.7}Attributes & \cellcolor[rgb]{0.7,0.7,0.7}Valid values & \cellcolor[rgb]{0.7,0.7,0.7}Description \\
		\hline
		Settings       & \textit{none} & \textit{none}       & General \NAME parameters can be set here. \\
		\hline
		OpenCL         & \textit{none} & \textit{none}       & OpenCL kernels to compute can be selected here. \\
		\hline
		Timing         & \textit{none} & \textit{none}       & Simulation time control directives are set here. \\
		\hline
		SPH            & \textit{none} & \textit{none}       & SPH method relevant data must be provided here. \\
		\hline
		Fluid          & n             & Number of particles & Fluid data must be provided in this section. \\
		               &               &                     & Each Fluid section will add new fluid to simulation. \\
		\hline
		Sensors        & \textit{none} & \textit{none}       & Sensor (or probes) can be specified in this section. \\
		\hline
		Movements      & \textit{none} & \textit{none}       & Motions to apply to the boundaries can be set here. \\
		\hline
		GhostParticles & \textit{none} & \textit{none}       & Ghost particles boundary can be set here. \\
		\hline
	\end{tabular}
	\caption{Valid section tags in the sphInput group.}
	\label{tables:caseSetup:XML:Groups}
\end{table}
%
\subsection{Settings group}
\label{sss:XML:Settings}
%
This group is used to set some \NAME general settings, that are not really
related to specific simulation. Table \ref{tables:caseSetup:Settings:Options}
shows all the valid options that can be set in this group.
%
\begin{table}[h!b!p!]\small
	\centering
	\begin{tabular}{| c | c | c | l | }
		\hline
		\cellcolor[rgb]{0.7,0.7,0.7}Tag & \cellcolor[rgb]{0.7,0.7,0.7}Attributes & \cellcolor[rgb]{0.7,0.7,0.7}Valid values & \cellcolor[rgb]{0.7,0.7,0.7}Description \\
		\hline
		Verbose & level & 0 & No verbose. \\
		        &       & 1 & Relevant data will be printed each frame. \\
		        &       & 2 & Relevant data will be printed each time step. \\
		\hline
		Device  & platform & id  & The OpenCL platform to use.\\
                & device   & id  & The device (inside the platform) to use. \\
                & type     & ALL & All the types of variables can be chosen. \\
                &          & CPU & Just CPU devices will be available. \\
                &          & GPU & Just GPU devices will be available. \\
                &          & ACCELERATOR & Just accelerators will be 
                available. \\
                &          & DEFAULT & Just the default OpenCL device will be 
                available. \\
		\hline
	\end{tabular}
	\caption{Valid settings option tags.}
	\label{tables:caseSetup:Settings:Options}
\end{table}
%
\subsection{OpenCL}
\label{sss:XML:OpenCL}
%
This group is used to set the OpenCL kernels that will be used to perform the
simulation. Table \ref{tables:caseSetup:OpenCL:Options} shows all the valid
options that can be set in this section.\rc
%
\NAME provides a version of each OpenCL kernel, but you can create custom
kernels in order to modify the software behaviour conveniently, with the
restriction that new kernels must preserve input parameters (sorted as the
original one).
%
\begin{table}[h!b!p!]\small
	\centering
	\begin{tabular}{| c | c | c | l | }
		\hline
		\cellcolor[rgb]{0.7,0.7,0.7}Tag & \cellcolor[rgb]{0.7,0.7,0.7}Attributes & \cellcolor[rgb]{0.7,0.7,0.7}Valid values & \cellcolor[rgb]{0.7,0.7,0.7}Description \\
		\hline
		Predictor & file & File path & Kernel file used at Leap-frog predictor stage. \\
		          &      &           & File extension (.cl) will appended automatically. \\
		\hline
		LinkList          & file & File path & Kernel file used to perform the Link-List, \\
		                  &      &           & i.e.- In-cell particles localization. \\
		                  &      &           & File extension (.cl) will appended automatically. \\
		\hline
		Rates             & file & File path & Kernel file used to perform particles interaction. \\
		                  &      &           & File extension (.cl) will appended automatically. \\
		\hline
		Corrector         & file & File path & Kernel file used at Leap-frog corrector stage. \\
		                  &      &           & File extension (.cl) will appended automatically. \\
		\hline
		TimeStep          & file & File path & Kernel file used to compute time step. \\
		                  &      &           & File extension (.cl) will appended automatically. \\
		\hline
		Reduction         & file & File path & Kernel file used to perform 
		reductions of variables \\
		                  &      &           & File extension (.cl) will appended automatically. \\
		\hline
		RadixSort         & file & File path & Kernel file used to sort particles by cell location. \\
		                  &      &           & File extension (.cl) will appended automatically. \\
		\hline
		Shepard           & file & File path & Kernel file used to perform 0th order correction. \\
		                  &      &           & File extension (.cl) will appended automatically. \\
		\hline
		Domain            & file & File path & Kernel file used to find and destroy particles out \\
		                  &      &           & of domain. \\
		                  &      &           & File extension (.cl) will appended automatically. \\
		\hline
		ElasticBounce     & file & File path & Kernel file used to perform simple wall boundary \\
		                  &      &           & condition. \\
		                  &      &           & File extension (.cl) will appended automatically. \\
		\hline
		DeLeffe           & file & File path & Kernel file used to perform DeLeffe wall boundary \\
		                  &      &           & condition. \\
		                  &      &           & File extension (.cl) will appended automatically. \\
		\hline
		DensInterpolation & file & File path & Kernel file used to perform density reinitialization. \\
		                  &      &           & File extension (.cl) will appended automatically. \\
		\hline
		Torque            & file & File path & Kernel file used to compute force and torque. \\
		                  &      &           & File extension (.cl) will appended automatically. \\
		\hline
		Bounds            & file & File path & Kernel file used to compute 
		fluid bounds. \\
		                  &      &           & File extension (.cl) will 
		                  appended automatically. \\
		\hline
		Energy            & file & File path & Kernel file used to compute 
		fluid energy. \\
		                  &      &           & File extension (.cl) will 
		                  appended automatically. \\
		\hline
		Portal            & file & File path & Kernel file used to perform particles teleporting. \\
		                  &      &           & File extension (.cl) will appended automatically. \\
		\hline
	\end{tabular}
	\caption{OpenCL kernels tags.}
	\label{tables:caseSetup:OpenCL:Options}
\end{table}
%
\subsection{Timing group}
\label{sss:XML:Timing}
%
This group is used to set simulation time stepping controls. Table
\ref{tables:caseSetup:Timing:Options} shows all the valid options
that can be set in this section.\rc
%
The minimum time step and the Courant factor are not affecting if the time 
step is not selected as ``Variable''.
%
\begin{table}[h!b!p!]\small
	\centering
	\begin{tabular}{| c | c | c | c | c | c | l | }
		\hline
		\cellcolor[rgb]{0.7,0.7,0.7}Tag & \cellcolor[rgb]{0.7,0.7,0.7}Attribute & \cellcolor[rgb]{0.7,0.7,0.7}Values & \cellcolor[rgb]{0.7,0.7,0.7}Attributes & \cellcolor[rgb]{0.7,0.7,0.7}Values & \cellcolor[rgb]{0.7,0.7,0.7}Attributes & \cellcolor[rgb]{0.7,0.7,0.7}Values \\
		\hline
		Option & name & Start &        &         & value  & Starting time. \\
		Option & name & Stop  & type   & Time    & value  & End time. \\
		       &      &                &         & Frames & value & Number of 
		       frames. \\
		       &      &                &        &         &       & (Can be 
		       mixed with time stop \\ 
		       &      &                &        &         &       & criteria) \\ 
		\hline
		       & name & Timestep       & value  & Variable &       & (Time 
		       step recomputed)\\
		       &      &                &        & Fix     & value & (Time step 
		       computed at start) \\
		       &      &                &        & time [s] & & \\
		\hline
		       & name & MinTimeStep    &        &         & value & Minimum 
		       allowed time step\\
		\hline
		       & name & Courant        &        &         & value & Courant 
		       factor\\
		\hline
		       & name & LogFile        & type   & No      &       & \\
		       &      &                &        & FPS     & value & Frames per second. \\
		       &      &                &        & IPF     & value & Iterations per frame. \\
		       &      &                &        &         &       & (Can be 
		       mixed with FPS) \\ 
		\hline
		       & name & EnFile        & type   & No      &       & \\
		       &      &                &        & FPS     & value & Frames per 
		       second. \\
		       &      &                &        & IPF     & value & Iterations 
		       per frame. \\
		       &      &                &        &         &       & (Can be 
		       mixed with FPS) \\ 
		\hline
		       & name & BoundsFile     & type   & No      &       & \\
		       &      &                &        & FPS     & value & Frames per 
		       second. \\
		       &      &                &        & IPF     & value & Iterations 
		       per frame. \\
		       &      &                &        &         &       & (Can be 
		       mixed with FPS) \\ 
		\hline
		       & name & Output         & type   & No      &       & \\
		       &      &                &        & FPS     & value & Frames per second. \\
		       &      &                &        & IPF     & value & Iterations per frame. \\
		       &      &                &        &         &       & (Can be 
		       mixed with FPS) \\ 
		\hline
	\end{tabular}
	\caption{Time control valid tags.}
	\label{tables:caseSetup:Timing:Options}
\end{table}
%
\subsection{SPH}
\label{sss:XML:SPH}
%
This group is used to set SPH method related settings. Table
\ref{tables:caseSetup:SPH:Options} shows all valid options that
can be set in this section.\rc
%
$\gamma$ value is a factor involved in the EOS
\ref{eq:governing_eqns:field_eqns:eos:arfm_2012}.\rc
%
Sound speed must be selected, at least, 10 times greater than the
maximum expected fluid velocity along the simulation. Sound speed
affects directly to the fluid compressibility, so higher sound
speed implies simulation improvements, but is too much computing
expensive (due to a time step reduction).\rc
%
In the equation \ref{eq:sph:discrete:timestep:dtmax} the time step
that must be selected is described (how to setup the time step is
described in the section \ref{sss:XML:Timing}), but in order to
reduce the Courant number you can setup a time step divisor
\textbf{DivDt}.\rc
%
\textbf{LLSteps} steps is the number of time steps between Link-List
computations, so a small performance can be win. For almost cases is
strongly recommended to let \textbf{LLSteps} = 1.\rc
%
The density interpolation (formerly density reinitialization) has been
described in the section \ref{ss:intro:aquagpusph:operators}. With
\textbf{DensSteps} option you may set the number of time steps that
will be computed before each density interpolation.\rc
%
The boundary conditions has been introduced in the section
\ref{ss:sph:discrete:BC}.\rc
%
Finally domain can be used to destroy the particles out of it,
transforming them into 0 mass fixed particles.
%
\begin{table}[h!b!p!]\small
	\centering
	\begin{tabular}{| c | c | c | c | c | }
		\hline
		\cellcolor[rgb]{0.7,0.7,0.7}Tag & \cellcolor[rgb]{0.7,0.7,0.7}Attribute & \cellcolor[rgb]{0.7,0.7,0.7}Values & \cellcolor[rgb]{0.7,0.7,0.7}Attributes & \cellcolor[rgb]{0.7,0.7,0.7}Values \\
		\hline
		Option & name & gamma              & value & $\gamma$ Batchelor 67 value \\
		\hline
		       &      & g                  & x     & $x$ vector component. \\
		       &      &                    & y     & $y$ vector component. \\
		       &      &                    & z     & $z$ vector component. \\
		\hline
		       &      & hfac               & value & $\frac{h}{dx}$ \\
		\hline
		       &      & deltar             & x     & $dx$ \\
		       &      &                    & y     & $dy$ \\
		       &      &                    & z     & $dz$ \\
		\hline
		       &      & cs                 & value & sound speed \\
		\hline
		       &      & DivDt              & value & $dt$ scale factor \\
		\hline
		       &      & refp               & value & $p_0$ \\
		\hline
		       &      & LLSteps            & value  & Number of steps \\
		\hline
		       &      & DensSteps          & value  & Number of steps \\
		       &      &                    &        & (0 to disable it) \\
		\hline
		       &      & Boundary           & value  & ElasticBounce \\
		       &      &                    &        & FixedParticles \\
		       &      &                    &        & DeLeffe \\
		\hline
		       &      & SlipCondition      & value  & NoSlip \\
		       &      &                    &        & FreeSlip \\
		\hline
		       &      & BoundDist          & value  & Minimum distance to wall. \\
		\hline
		       &      & BoundElasticFactor & value  & Elastic interaction factor. \\
		\hline
		       &      & Shepard            & value  & None \\
		       &      &                    &        & Force \\
		       &      &                    &        & Dens \\
		       &      &                    &        & ForceDens \\
		\hline
		       &      & Domain             & x      & Starting $x$ coordinate \\
		       &      &                    & y      & Starting $y$ coordinate \\
		       &      &                    & z      & Starting $y$ coordinate \\
		       &      &                    & l      & Length \\
		       &      &                    & b      & Breadth \\
		       &      &                    & h      & Height \\
		\hline
	\end{tabular}
	\caption{SPH method available options.}
	\label{tables:caseSetup:SPH:Options}
\end{table}
%
\subsection{Fluid}
\label{sss:XML:Fluid}
%
This group is used to add fluids, i.e. each instance of this group
will be considered as a new fluid. \textbf{Fluid} tag must have the
attribute \textbf{n} specifying the total number of particles and
vertices involved. Table \ref{tables:caseSetup:Fluid:Options} shows
all valid options that can be set in this section.\rc
%
$\mu$ must be the real dynamic viscosity of the fluid, but since
weakly compressible SPH is a purely explicit method, will turn
unstable if the $\alpha$ value don't reach a minimum value, $\mu$ can
be corrected depending on $\alpha$ value set \footnote{Minimum
suggested value is 0.01, but is recommended to get 0.03 at least}.\rc
%
Regarding the particles file allowed formats are discussed in the
section \ref{ss:partsFile}.
%
\begin{table}[h!b!p!]\small
	\centering
	\begin{tabular}{| c | c | c | c | c | }
		\hline
		\cellcolor[rgb]{0.7,0.7,0.7}Tag & \cellcolor[rgb]{0.7,0.7,0.7}Attribute & \cellcolor[rgb]{0.7,0.7,0.7}Values & \cellcolor[rgb]{0.7,0.7,0.7}Attributes & \cellcolor[rgb]{0.7,0.7,0.7}Values \\
		\hline
		Option & name & gamma              & value & $\gamma$ Batchelor 67 value \\
		\hline
		       &      & refd               & value & $\rho_0$ \\
		\hline
		       &      & Viscdyn            & value & $\mu$ \\
		\hline
		       &      & alpha              & value & $\alpha \geq 8 \frac{\mu_{sim}}{\rho c_s h} $ \\
		\hline
		Load   & file & Fluid particles and vertices file. & & \\
		\hline
	\end{tabular}
	\caption{Fluid generation options.}
	\label{tables:caseSetup:Fluid:Options}
\end{table}
%
\subsection{Sensors}
\label{sss:XML:Sensors}
%
This group is used to set sensors. Sensors interpolates and print several
fields on the selected point, naming:
%
\begin{enumerate}
	\item \textbf{Position}: Sensor position.
	\item \textbf{Pressure}: Pressure $[\mbox{Pa}]$.
	\item \textbf{Density}: Density $[\mbox{kg/m}^3]$.
	\item \textbf{Kernel completion}: $\sum\limits_{b \in \mathrm{fluid}} \frac{W_{ab}}{\rho_b}m_b$.
\end{enumerate}
%
You can add as many sensors as you want adding several instances of the
\textbf{Sensor} tag. \NAME provides an OpenCL kernel for sensors, but you can
optionally change it for your own version in order to change interpolation
method (for instance MLS as described by \citet{colagrossi2003}). Provided
version of OpenCL kernel can interpolate the value on usual SPH way (using
Shepard correction as is described in section \ref{ss:running:sensorsoutput}),
or return maximum values detected.
%
\begin{table}[h!b!p!]\small
	\centering
	\begin{tabular}{| c | c | c | }
		\hline
		\cellcolor[rgb]{0.7,0.7,0.7}Tag & \cellcolor[rgb]{0.7,0.7,0.7}Attribute & \cellcolor[rgb]{0.7,0.7,0.7}Values \\
		\hline
		FPS    & value & Output frames per second \\
		\hline
		Script & file  & OpenCL kernel file \\
		\hline
		Sensor & x     & $x$ coordinate \\
		\hline
		       & y     & $y$ coordinate \\
		\hline
		       & z     & $z$ coordinate \\
		\hline
		       & type  & Interpolated \\
		       &       & Maximum \\
		\hline
	\end{tabular}
	\caption{Sensors set options.}
	\label{tables:caseSetup:Sensors:Options}
\end{table}
%
\subsection{Movements}
\label{sss:XML:Movements}
%
This group is used to set motions. Movements will be used to set fixed
particles, DeLeffe area elements, and Ghost particles wall corners motion
along the simulation time. You can add as many motions as you want adding
several instances of the \textbf{Movement} tag, and final motion will be
computed as the superposition of them. \rc
%
Each added movement must have its own XML definition file. The data that
must be provided in this file depends on the movement selected type.
Valid types of movement are:
%
\begin{enumerate}
	\item \textbf{LIQuaternion}: In this movement you must provide a data
	table file containing the solid quaternion at several time instants
	along the simulation. Quaternion is defined by center and axes, i.e.
	in 2D you must provide at least the center and the x axis, and in 3D
	the center and the x and y axes. Quaternion at each time step will be
	linearly interpolated from the provided data table.
	\item \textbf{ScriptQuaternion}: This movement requires a Python script
	that will be executed each time step in order to receive the quaternion.
\end{enumerate}
%
Quaternion based motions are usually the best choice because they provide
full control of the motion, but has the disadvantage that will overwrite
all the previous computed motions, so motions superposition is not supported.\rc
%
Actually only Quaternion based motions are supported.\rc
%
See examples chapter \ref{s:examples} in order to know how the motion definition
file must be written for each type of movement.
%
\begin{table}[h!b!p!]\small
	\centering
	\begin{tabular}{| c | c | c | }
		\hline
		\cellcolor[rgb]{0.7,0.7,0.7}Tag & \cellcolor[rgb]{0.7,0.7,0.7}Attribute & \cellcolor[rgb]{0.7,0.7,0.7}Values \\
		\hline
		Movement & type & LIQuaternion \\
		\hline
		         &      & ScriptQuaternion \\
		\hline
		         & file & XML definition file \\
		\hline
	\end{tabular}
	\caption{Movements set options.}
	\label{tables:caseSetup:Movements:Options}
\end{table}
%
\subsection{GhostParticles}
\label{sss:XML:GhostParticles}
%
This group is used to set Ghost particles boundary condition. Ghost particles
will be discussed on section \ref{sss:aquagpusph:boundaries:ghostparticles}.\rc
%
In the table \ref{tables:caseSetup:GhostParticles:Options} all the valid options
for Ghost particles are shown. Default values are `SSM' (symmetric model) for the
pressure and the tangent velocity, and `ASM' (antisymmetric model) for the normal
velocity.\rc
%
\textit{TangentUModel} can be selected in order to impose free-slip or no-slip.
Free-slip condition is imposed with a `SSM' model, and being consistent for a
large number of cases, and no-slip is imposed with `ASM' or with `Takeda' (not
implemented yet).\rc
%
The consequences of selecting each mirroring model for each fields can be found
in the reference \citep{MaciaetalPTP}.\rc
%
In the table \ref{tables:caseSetup:GhostParticles:Groups} the subgroups that can
be used are listed. Ghost particles algorithm is quite different of Fix particles
or boundary integrals, and require to set the walls separately, but not within the
particles definition, so it's strongly recommended not using ghost particles with
large number of solid boundary walls.\rc
%
In 3D cases, each wall must have 3 or 4 (triangular or quadrangular walls
respectively) `Vertex' tags, with the \textit{x}, \textit{y}, \textit{z} coordinates
attributes, and in 2D simulations 2 `Vertex' tags must be provided with the
\textit{x}, \textit{y} coordinates attributes.
%
\begin{table}[h!b!p!]\small
	\centering
	\begin{tabular}{| c | c | c | }
		\hline
		\cellcolor[rgb]{0.7,0.7,0.7}Tag & \cellcolor[rgb]{0.7,0.7,0.7}Attribute & \cellcolor[rgb]{0.7,0.7,0.7}Values \\
		\hline
		PressModel    & value & ASM \\
		\hline
		              &       & SSM \\
		\hline
		              &       & Takeda \\
		\hline
		NormalUModel  & value & ASM \\
		\hline
		              &       & SSM \\
		\hline
		              &       & Takeda \\
		\hline
		              &       & U0M \\
		\hline
		TangentUModel & value & ASM \\
		\hline
		              &       & SSM \\
		\hline
		              &       & Takeda \\
		\hline
		              &       & U0M \\
		\hline
	\end{tabular}
	\caption{Ghost particles options.}
	\label{tables:caseSetup:GhostParticles:Options}
\end{table}
%
\begin{table}[h!b!p!]\small
	\centering
	\begin{tabular}{| c | c | c | }
		\hline
		\cellcolor[rgb]{0.7,0.7,0.7}Subgroup & \cellcolor[rgb]{0.7,0.7,0.7}Description \\
		\hline
		Wall & Wall to mirror the fluid particles using set models. \\
		\hline
	\end{tabular}
	\caption{Ghost particles options.}
	\label{tables:caseSetup:GhostParticles:Groups}
\end{table}
%
